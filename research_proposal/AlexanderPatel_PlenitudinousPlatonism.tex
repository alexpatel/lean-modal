\documentclass[12pt]{article}
\usepackage{graphicx}
\usepackage{latexsym}
\usepackage{amssymb,amsmath}
\usepackage{mathtools}
\usepackage{amsthm}
\usepackage{etoolbox}
\usepackage{hyperref}
\usepackage[margin=1.0in]{geometry}
\allowdisplaybreaks

\theoremstyle{definition}
\newtheorem{theorem}{Theorem}
\newtheorem{defn}{Definition}
\newtheorem{prop}{Proposition}
\newtheorem{cor}{Corollary}
\newtheorem{lemma}[theorem]{Lemma}

\usepackage[T1]{fontenc}
\setlength\parindent{0pt}
\usepackage[parfill]{parskip}
\patchcmd{\thebibliography}{\section*}{\section}{}{}

\begin{document}

\begin{center}
\textbf{On \textit{plenitudinous platonism}} \\
\textbf{Research Proposal} \\
24.711 Topics in Philosophical Logic \\
Alexander H. Patel \\
{\tt alexanderpatel@college.harvard.edu} \\
\today
\end{center}

\begin{quotation}
"The demand is not to be denied: every jump must be barred from our deductions.
That it is so hard to satisfy must be set down to the tediousness of proceeding
step by step. Every proof which is even a little complicated threatens to
become inordinately long. And moreover, the excessive variety of logical forms
that has gone into the shaping of our language makes it difficult to isolate a
set of modes of inference which is both sufficient to cope with all cases and
easy to take in at a glance. To minimize these drawbacks, I invented my concept
writing. [\textit{Begriffsschrift}]" (Gottlob Frege, \textit{Foundations of
Arithmetic}, p. VI)
\end{quotation}


\tableofcontents

\section{Draft Note}

This document is structured as a proposal for a research project, the output of
which will be a final term paper for 24.711. My intention is to take an
interdisciplinary approach and rebut an argument in the philosophy of
mathematics using the tools of formal verification from computer science.
Formal verification is the proving or disproving of the correctness of
algorithms underlying the behavior of a software or hardware system.

I will be working as a research assistant from June 5th until August 27th for
Professor Margo Seltzer in the Harvard Computer Science department. The topic
of her group's research is the synthesis (automatic programming) and
verification of operating system kernel components. Because my 24.711 project
involves research and programming within the same topic, my intention is to
develop the final paper in tandem with my work in this lab because I do not
have experience yet with the appropriate software tools. I expect this paper
will evolve into my writing sample for PhD applications in philosophy and lead
into my undergraduate thesis in philosophy and mathematics (joint
concentration).

I plan to meet with Vann once a month, beginning at the start of June and
ending before September 10th when the paper is due (in total four times).  The
objective of these meetings is to present my research findings and solicit
direction for the project.

\section{Research Topic}

Platonism in the philosophy of mathematics defends that mathematical objects
are human-independent abstract entities.  \textit{plenitudinous platonism} (hereby PP, also known as \textit{full-blooded platonism}), first proposed by Mark Balaguer in
\textit{Platonism and Anti-Platonism in Mathematics}, is a version of
mathematical platonism that holds that any mathematical object that is
logically possible, necessarily exists. As a consequence, according to this
doctrine every consistent theory of mathematics refers necessarily to set of
existent abstract entities \cite{balaguer}.

Balaguer argues in his book that PP is the only formulation that resolves both
of Paul Benacceraf's criticims of platonism: that mathematical knowledge is
impossible because mathematical objects are causally inert, and that platonism
requires a commitment to unique mathematical objects and these objects are
provably non-unique.  Because in PP every consistent mathematical theory
describes a part of the mathematical universe, our beliefs about mathematical
objects consistute knowledge of those objects and so the epistemic gap is
bridged. Second, Balaguer argues that PP resolves Benacerraf's non-uniqueness
objection because it avoids the standard platonistic commitment to the
uniqueness of mathematical objects because (all consistent theories manifest)
\cite{review}.

Balauger offers in his book a partial formulation of PP in second-order modal
logic:

\begin{equation}
(\exists x)(Mx) \& (Y)[\Diamond (\exists x) (Mx \& Yx) \rightarrow (\exists x)(Mx \& Yx)]
\end{equation}

Here, $x$ is a first-order variable, $Y$ is a second-order variable, and $Mx$
means '$x$ is a mathematical object'.

In "Just What is Full-Blooded Platonism?", Greg Restall argues that Balaguer's
second-order formulation of PP yields a contradiction. From (1), he argues that one can derive:

\begin{equation}
    p \supset \neg \Diamond (\exists x) (Mx \& \neg p)
\end{equation}

Informally, (2) states that if proposition $p$ is true then it is impossible
for there to be any mathematical objects and for $\neg p$ to be true. This
result is devastating for Balaguer's project, according to Restall. For
example, if $p$ is 'Cherry took a nap today' and Cherry did take a nap today,
then we would conclude that it is not possible that there are both mathematical
objects and that Cherry did not nap today, which is an undesirable
result\cite{restall}.

Restall argues that the formalization of PP ought then to be expressed in a
third-order modal logic so as to restrict the scope of $Y$ to mathematical
properties alone. According to Restall, this third-order formalization yields
similar undesirable results.

\section{Research Objective}

The objective of this research project first is to write a computer program
formal verification in a logic programming language of Restall's objections
against formalized PP.

By constructing this computer program so that it may be embedded in my term
paper through transpilation (translation) into LaTeX, my second intention is to
present a philosophy paper the contents of which are verified to be sound by
virtue of having successfully compiled its LaTeX source code.

\section{Research Output}

The output will be a 15-20 page term paper due on September 10th, 2017. Any
additional source code that is required to generate or compile this paper will
be submitted in a public code repository.

\section{Project Components}

\textbf{Modal Logic Programming Languages}

There are many programming languages and software tools for formal verification
and both automated theorem proving, which require no human interaction other
than the specification of a goal, and assisted theorem proving, which assists a
human user by "filling in the inferential blanks", per se.. 

Such tools are used widely in the technology industry for tasks such as
verification of hardware correctness, cryptographic systems, and digital
circuits. Such languages include Coq, Isabelle, HOL ("Higher-Order Logic"), and
ACL2. Most of these tools are written in ML or Prolog. 

I have included here a list of such tools that may serve useful for verifying
the language used by Restall and Balaguer; the utility of a language ultimately
will be found in the fruitfulness and efficiency of the logic proofs I am able
to verify with it.

\begin{itemize}
    \item Isabelle/HOL (\texttt{https://isabelle.in.tum.de/}) - an interactive
        proof assistant written in Standard ML. Isabelle natively contains a
        library for working with modal logics, it is unclear yet whether this
        supports higher-order logics (although Isabelle/HOL certainly does this
        very well) \cite{modalisabelle}.
    \item Prolog (\texttt{http://www.swi-prolog.org/})- Prolog is a
        general-purpose logic programming language widely used in industry.
        It's inference engine is capable of capturing and validating complex
        relationships and rule-based logical queries.
    \item MOLTAP (\texttt{http://twan.home.fmf.nl/moltap/related-work.html}) -
        an automated theorem prover for modal logic that can both construct
        deductions and generate Kripke models are reputations. This library
        does not natively support higher-order modal logics.
    \item ModLeanTAP  (\texttt{http://formal.iti.kit.edu/beckert/modlean/}) -
        written in Prolog, ModLeanTAP is a tableau calculus for propositional
        modal logics. It does not appear that this supports higher-order modal
        logics natively.

\end{itemize}

\textbf{Transpilation into LaTeX}

Transpilation is the process of translating a computer program from one
language to another. Transpiling a formal verification from a logic programming
language into LaTeX appears at first glance to entail writing a mapping from
strings and syntax rules in the programming language to LaTeX markup. However,
proof assistants may condense the longer inferences in a given program or
otherwise obfuscate the raw proofs, and so additional work will be needed here.

Isabelle/HOL contains a robust system for outputting proofs constructed within
the Isabelle proof assistant to LaTeX. It is doubtful that this library would
natively support the typesetting necessary to transpile modal logic proofs, and
so such functionality would have to be built on top of the existing
transpilation tools. Based on my understanding of the structure of Prolog,
writing a transpiler for Prolog proofs is a well-defined and solved problem.

\section{Research Questions}

\begin{enumerate}
    \item Is the philosophy of mathematics amenable to further formalization?
        What kinds of formalizations are needed to add further rigor to the
        landmark positions of this field?
    \item Are existing formal verification tools and proof assistants equipped
        to express the content of arguments in the philosophy of mathematics?
    \item If existing tools are not expressive enough, can they be extended so
        as to be able to capture the complex inferences and abstractions
        inherent to the philosophy of mathematics?
    \item Can higher-order modal logic claims be expressed in a logic
        programming language? Can sentences in higher-order modal logic be
        reduced to sentences in first-order modal logic or the predicate
        calculus? How computationally expensive is it to translate these
        sentences for a theorem prover be useful for?
    \item Is it only possible to provide formal verifications of positive
        claims in higher-order model logic (syntactic proofs in the formal
        language), or are there tools that can refute arguments by
        counter-example (semantic proofs via Kripke frames)? The former of
        these is called \textit{deductive verification} while the latter is
        called \textit{model checking}.
    \item Is Greg Restall correct that the formalization of Plenituduous
        Platonism presented by Balaguer yields a contradiction? Does Restall's
        recommendations for amending this formalization also fail, as he claims?
\end{enumerate}

\section{Timeline}
    \begin{center}
        \begin{tabular}[t]{|c|p{4in}|}
        \hline
        \textit{Date} & \textit{Goal} \\ [0.5ex] \hline\hline
        Early June & Develop research proposal and hone plan with Vann. \\ \hline
        Early July & Investigate assistant and automated theorem provers for
            higher-order modal logic; complete language tutorials and implement
            example proofs;  \\  \hline
        Early August & Complete first pass formal verification of Restall's and
            Balaguer's dispute; figure out how to transpile proofs into LaTeX.
            \\ \hline
        Late August/Early September & Paper draft review with Vann. Final term
        paper (15-20 pages) due September 10, 2017. \\ [1ex] \hline
    \end{tabular}
    \end{center}

\begin{thebibliography}{9}
\bibitem{balaguer} 
Balaguer, Mark.
\textit{Platonism and Anti-Platonism in Mathematics}. 
Oxford University Press, New York, 1998.
 
\bibitem{restall} 
Restall, Greg. 
"Just what is Full-Blooded Platonism?"
Philosophia Mathematica 11 (1):82--91 (2003).
 
\bibitem{modaltemporal} 
Orgun, Mehmet A. and Wanli Ma.
"An Overview of Temporal and Modal Logic Programming".
International Conference on Temporal Logic 1994.
\\\texttt{http://www-cs-faculty.stanford.edu/\~{}uno/abcde.html}

\bibitem{naturalized}
Linsky, Bernard and Edward N. Zalta.
"Naturalized Platonism versus Platonized Naturalism".
\textit{The Journal of Philosophy}, Vol. 92, No. 10 (Oct., 1995), pp. 525-555
\\\texttt{http://www.jstor.org/stable/2940786}

\bibitem{review}
Colyvan, Mark and Edward N. Zalta. 
"Mathematics: Truth and Fiction?" 
Philosophia Mathematica (1999) 7 (3): 336-349.
\\\texttt{https://mally.stanford.edu/Papers/balaguer.pdf}

\bibitem{formalver}
Avigad, Jeremy.
"Understanding, formal verification, and the philosophy of mathematics".
Journal of the Indian Council of Philosophical Research, 27, 161-197
\\\texttt{http://repository.cmu.edu/philosophy/616/}

\bibitem{isabelletut}
Nipkow et al. 
"Isabelle/HOL: A Proof Assistant for Higher-Order Logic".
Springer-Verlag, Berlin, 2016.
\\\texttt{http://isabelle.in.tum.de/doc/tutorial.pdf}

\bibitem{modalcs}
Lambert, Leigh.
"Modal Logic in Computer Science".
\\\texttt{http://www.cs.brandeis.edu/~cs112/cs112-2004/newReadings/ModalLogicInCS.pdf}

\bibitem{modalprovers}
Benzm\"uller et al.
"Implementing and Evaluating Provers for First-Order Modal Logics"
ECAI 2012.
\\\texttt{https://page.mi.fu-berlin.de/cbenzmueller/papers/C34.pdf}

\bibitem{isabellesys}
Wenzel, Makarius.
"The Isabelle System Manual".
12 December 2016.
\\\texttt{http://isabelle.in.tum.de/doc/system.pdf}

\bibitem{modalisabelle}
Isabelle Source Code: Sequents and Modals
\\\texttt{https://github.com/seL4/isabelle/tree/master/src/Sequents}

\end{thebibliography}

\end{document}
